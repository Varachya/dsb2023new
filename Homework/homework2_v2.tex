% Options for packages loaded elsewhere
\PassOptionsToPackage{unicode}{hyperref}
\PassOptionsToPackage{hyphens}{url}
%
\documentclass[
]{article}
\usepackage{amsmath,amssymb}
\usepackage{lmodern}
\usepackage{iftex}
\ifPDFTeX
  \usepackage[T1]{fontenc}
  \usepackage[utf8]{inputenc}
  \usepackage{textcomp} % provide euro and other symbols
\else % if luatex or xetex
  \usepackage{unicode-math}
  \defaultfontfeatures{Scale=MatchLowercase}
  \defaultfontfeatures[\rmfamily]{Ligatures=TeX,Scale=1}
\fi
% Use upquote if available, for straight quotes in verbatim environments
\IfFileExists{upquote.sty}{\usepackage{upquote}}{}
\IfFileExists{microtype.sty}{% use microtype if available
  \usepackage[]{microtype}
  \UseMicrotypeSet[protrusion]{basicmath} % disable protrusion for tt fonts
}{}
\makeatletter
\@ifundefined{KOMAClassName}{% if non-KOMA class
  \IfFileExists{parskip.sty}{%
    \usepackage{parskip}
  }{% else
    \setlength{\parindent}{0pt}
    \setlength{\parskip}{6pt plus 2pt minus 1pt}}
}{% if KOMA class
  \KOMAoptions{parskip=half}}
\makeatother
\usepackage{xcolor}
\usepackage[margin=1in]{geometry}
\usepackage{color}
\usepackage{fancyvrb}
\newcommand{\VerbBar}{|}
\newcommand{\VERB}{\Verb[commandchars=\\\{\}]}
\DefineVerbatimEnvironment{Highlighting}{Verbatim}{commandchars=\\\{\}}
% Add ',fontsize=\small' for more characters per line
\usepackage{framed}
\definecolor{shadecolor}{RGB}{248,248,248}
\newenvironment{Shaded}{\begin{snugshade}}{\end{snugshade}}
\newcommand{\AlertTok}[1]{\textcolor[rgb]{0.94,0.16,0.16}{#1}}
\newcommand{\AnnotationTok}[1]{\textcolor[rgb]{0.56,0.35,0.01}{\textbf{\textit{#1}}}}
\newcommand{\AttributeTok}[1]{\textcolor[rgb]{0.77,0.63,0.00}{#1}}
\newcommand{\BaseNTok}[1]{\textcolor[rgb]{0.00,0.00,0.81}{#1}}
\newcommand{\BuiltInTok}[1]{#1}
\newcommand{\CharTok}[1]{\textcolor[rgb]{0.31,0.60,0.02}{#1}}
\newcommand{\CommentTok}[1]{\textcolor[rgb]{0.56,0.35,0.01}{\textit{#1}}}
\newcommand{\CommentVarTok}[1]{\textcolor[rgb]{0.56,0.35,0.01}{\textbf{\textit{#1}}}}
\newcommand{\ConstantTok}[1]{\textcolor[rgb]{0.00,0.00,0.00}{#1}}
\newcommand{\ControlFlowTok}[1]{\textcolor[rgb]{0.13,0.29,0.53}{\textbf{#1}}}
\newcommand{\DataTypeTok}[1]{\textcolor[rgb]{0.13,0.29,0.53}{#1}}
\newcommand{\DecValTok}[1]{\textcolor[rgb]{0.00,0.00,0.81}{#1}}
\newcommand{\DocumentationTok}[1]{\textcolor[rgb]{0.56,0.35,0.01}{\textbf{\textit{#1}}}}
\newcommand{\ErrorTok}[1]{\textcolor[rgb]{0.64,0.00,0.00}{\textbf{#1}}}
\newcommand{\ExtensionTok}[1]{#1}
\newcommand{\FloatTok}[1]{\textcolor[rgb]{0.00,0.00,0.81}{#1}}
\newcommand{\FunctionTok}[1]{\textcolor[rgb]{0.00,0.00,0.00}{#1}}
\newcommand{\ImportTok}[1]{#1}
\newcommand{\InformationTok}[1]{\textcolor[rgb]{0.56,0.35,0.01}{\textbf{\textit{#1}}}}
\newcommand{\KeywordTok}[1]{\textcolor[rgb]{0.13,0.29,0.53}{\textbf{#1}}}
\newcommand{\NormalTok}[1]{#1}
\newcommand{\OperatorTok}[1]{\textcolor[rgb]{0.81,0.36,0.00}{\textbf{#1}}}
\newcommand{\OtherTok}[1]{\textcolor[rgb]{0.56,0.35,0.01}{#1}}
\newcommand{\PreprocessorTok}[1]{\textcolor[rgb]{0.56,0.35,0.01}{\textit{#1}}}
\newcommand{\RegionMarkerTok}[1]{#1}
\newcommand{\SpecialCharTok}[1]{\textcolor[rgb]{0.00,0.00,0.00}{#1}}
\newcommand{\SpecialStringTok}[1]{\textcolor[rgb]{0.31,0.60,0.02}{#1}}
\newcommand{\StringTok}[1]{\textcolor[rgb]{0.31,0.60,0.02}{#1}}
\newcommand{\VariableTok}[1]{\textcolor[rgb]{0.00,0.00,0.00}{#1}}
\newcommand{\VerbatimStringTok}[1]{\textcolor[rgb]{0.31,0.60,0.02}{#1}}
\newcommand{\WarningTok}[1]{\textcolor[rgb]{0.56,0.35,0.01}{\textbf{\textit{#1}}}}
\usepackage{longtable,booktabs,array}
\usepackage{calc} % for calculating minipage widths
% Correct order of tables after \paragraph or \subparagraph
\usepackage{etoolbox}
\makeatletter
\patchcmd\longtable{\par}{\if@noskipsec\mbox{}\fi\par}{}{}
\makeatother
% Allow footnotes in longtable head/foot
\IfFileExists{footnotehyper.sty}{\usepackage{footnotehyper}}{\usepackage{footnote}}
\makesavenoteenv{longtable}
\usepackage{graphicx}
\makeatletter
\def\maxwidth{\ifdim\Gin@nat@width>\linewidth\linewidth\else\Gin@nat@width\fi}
\def\maxheight{\ifdim\Gin@nat@height>\textheight\textheight\else\Gin@nat@height\fi}
\makeatother
% Scale images if necessary, so that they will not overflow the page
% margins by default, and it is still possible to overwrite the defaults
% using explicit options in \includegraphics[width, height, ...]{}
\setkeys{Gin}{width=\maxwidth,height=\maxheight,keepaspectratio}
% Set default figure placement to htbp
\makeatletter
\def\fps@figure{htbp}
\makeatother
\setlength{\emergencystretch}{3em} % prevent overfull lines
\providecommand{\tightlist}{%
  \setlength{\itemsep}{0pt}\setlength{\parskip}{0pt}}
\setcounter{secnumdepth}{-\maxdimen} % remove section numbering
\ifLuaTeX
  \usepackage{selnolig}  % disable illegal ligatures
\fi
\IfFileExists{bookmark.sty}{\usepackage{bookmark}}{\usepackage{hyperref}}
\IfFileExists{xurl.sty}{\usepackage{xurl}}{} % add URL line breaks if available
\urlstyle{same} % disable monospaced font for URLs
\hypersetup{
  pdftitle={Homework 2},
  pdfauthor={Your name goes here},
  hidelinks,
  pdfcreator={LaTeX via pandoc}}

\title{Homework 2}
\author{Your name goes here}
\date{2023-05-22}

\begin{document}
\maketitle

{
\setcounter{tocdepth}{2}
\tableofcontents
}
\begin{center}\rule{0.5\linewidth}{0.5pt}\end{center}

\hypertarget{data-visualisation---exploration}{%
\section{Data Visualisation -
Exploration}\label{data-visualisation---exploration}}

Now that you've demonstrated your software is setup, and you have the
basics of data manipulation, the goal of this assignment is to practice
transforming, visualising, and exploring data.

\hypertarget{mass-shootings-in-the-us}{%
\section{Mass shootings in the US}\label{mass-shootings-in-the-us}}

In July 2012, in the aftermath of a mass shooting in a movie theater in
Aurora, Colorado,
\href{https://www.motherjones.com/politics/2012/07/mass-shootings-map/}{Mother
Jones} published a report on mass shootings in the United States since
1982. Importantly, they provided the underlying data set as
\href{https://www.motherjones.com/politics/2012/12/mass-shootings-mother-jones-full-data/}{an
open-source database} for anyone interested in studying and
understanding this criminal behavior.

\hypertarget{obtain-the-data}{%
\subsection{Obtain the data}\label{obtain-the-data}}

\begin{verbatim}
## Rows: 125
## Columns: 14
## $ case                 <chr> "Oxford High School shooting", "San Jose VTA shoo~
## $ year                 <dbl> 2021, 2021, 2021, 2021, 2021, 2021, 2020, 2020, 2~
## $ month                <chr> "Nov", "May", "Apr", "Mar", "Mar", "Mar", "Mar", ~
## $ day                  <dbl> 30, 26, 15, 31, 22, 16, 16, 26, 10, 6, 31, 4, 3, ~
## $ location             <chr> "Oxford, Michigan", "San Jose, California", "Indi~
## $ summary              <chr> "Ethan Crumbley, a 15-year-old student at Oxford ~
## $ fatalities           <dbl> 4, 9, 8, 4, 10, 8, 4, 5, 4, 3, 7, 9, 22, 3, 12, 5~
## $ injured              <dbl> 7, 0, 7, 1, 0, 1, 0, 0, 3, 8, 25, 27, 26, 12, 4, ~
## $ total_victims        <dbl> 11, 9, 15, 5, 10, 9, 4, 5, 7, 11, 32, 36, 48, 15,~
## $ location_type        <chr> "School", "Workplace", "Workplace", "Workplace", ~
## $ male                 <lgl> TRUE, TRUE, TRUE, TRUE, TRUE, TRUE, TRUE, TRUE, T~
## $ age_of_shooter       <dbl> 15, 57, 19, NA, 21, 21, 31, 51, NA, NA, 36, 24, 2~
## $ race                 <chr> NA, NA, "White", NA, NA, "White", NA, "Black", "B~
## $ prior_mental_illness <chr> NA, "Yes", "Yes", NA, "Yes", NA, NA, NA, NA, NA, ~
\end{verbatim}

\begin{longtable}[]{@{}
  >{\raggedright\arraybackslash}p{(\columnwidth - 2\tabcolsep) * \real{0.3611}}
  >{\raggedright\arraybackslash}p{(\columnwidth - 2\tabcolsep) * \real{0.6389}}@{}}
\toprule()
\begin{minipage}[b]{\linewidth}\raggedright
column(variable)
\end{minipage} & \begin{minipage}[b]{\linewidth}\raggedright
description
\end{minipage} \\
\midrule()
\endhead
case & short name of incident \\
year, month, day & year, month, day in which the shooting occurred \\
location & city and state where the shooting occcurred \\
summary & brief description of the incident \\
fatalities & Number of fatalities in the incident, excluding the
shooter \\
injured & Number of injured, non-fatal victims in the incident,
excluding the shooter \\
total\_victims & number of total victims in the incident, excluding the
shooter \\
location\_type & generic location in which the shooting took place \\
male & logical value, indicating whether the shooter was male \\
age\_of\_shooter & age of the shooter when the incident occured \\
race & race of the shooter \\
prior\_mental\_illness & did the shooter show evidence of mental illness
prior to the incident? \\
\bottomrule()
\end{longtable}

\hypertarget{explore-the-data}{%
\subsection{Explore the data}\label{explore-the-data}}

\hypertarget{specific-questions}{%
\subsubsection{Specific questions}\label{specific-questions}}

\begin{itemize}
\tightlist
\item
  Generate a data frame that summarizes the number of mass shootings per
  year.
\end{itemize}

\begin{Shaded}
\begin{Highlighting}[]
\NormalTok{mass\_shootings }\SpecialCharTok{\%\textgreater{}\%}
  \FunctionTok{group\_by}\NormalTok{(year) }\SpecialCharTok{\%\textgreater{}\%}
  \FunctionTok{summarise}\NormalTok{(}\AttributeTok{count=}\FunctionTok{n}\NormalTok{())}
\end{Highlighting}
\end{Shaded}

\begin{verbatim}
## # A tibble: 37 x 2
##     year count
##    <dbl> <int>
##  1  1982     1
##  2  1984     2
##  3  1986     1
##  4  1987     1
##  5  1988     1
##  6  1989     2
##  7  1990     1
##  8  1991     3
##  9  1992     2
## 10  1993     4
## # i 27 more rows
\end{verbatim}

\begin{itemize}
\tightlist
\item
  Generate a bar chart that identifies the number of mass shooters
  associated with each race category. The bars should be sorted from
  highest to lowest and each bar should show its number.
\end{itemize}

\begin{Shaded}
\begin{Highlighting}[]
\NormalTok{race\_mass\_shootings}\OtherTok{\textless{}{-}}\NormalTok{mass\_shootings }\SpecialCharTok{\%\textgreater{}\%}
  \FunctionTok{group\_by}\NormalTok{(race) }\SpecialCharTok{\%\textgreater{}\%} 
  \FunctionTok{summarise}\NormalTok{(}\AttributeTok{count=}\FunctionTok{n}\NormalTok{()) }\SpecialCharTok{\%\textgreater{}\%} 
  \FunctionTok{arrange}\NormalTok{(}\FunctionTok{desc}\NormalTok{(count)) }\SpecialCharTok{\%\textgreater{}\%} 
  \FunctionTok{drop\_na}\NormalTok{() }\SpecialCharTok{\%\textgreater{}\%} 
  \FunctionTok{ggplot}\NormalTok{(}\FunctionTok{aes}\NormalTok{(}\AttributeTok{x=}\FunctionTok{reorder}\NormalTok{(race,}\SpecialCharTok{{-}}\NormalTok{count),}\AttributeTok{y=}\NormalTok{count))}\SpecialCharTok{+}
  \FunctionTok{geom\_col}\NormalTok{()}\SpecialCharTok{+} 
  \FunctionTok{geom\_text}\NormalTok{(}\FunctionTok{aes}\NormalTok{(}\AttributeTok{label=}\NormalTok{count),}\AttributeTok{vjust=}\SpecialCharTok{{-}}\FloatTok{0.5}\NormalTok{) }\SpecialCharTok{+}
  \FunctionTok{labs}\NormalTok{(}\AttributeTok{title =} \StringTok{"Mass Shootings by race"}\NormalTok{,}
       \AttributeTok{x =} \StringTok{"Race"}\NormalTok{, }\AttributeTok{y=}\StringTok{"Number of Mass Shootings"}\NormalTok{) }\SpecialCharTok{+}
  \FunctionTok{theme\_bw}\NormalTok{()}

\FunctionTok{print}\NormalTok{(race\_mass\_shootings)}
\end{Highlighting}
\end{Shaded}

\includegraphics{homework2_v2_files/figure-latex/unnamed-chunk-4-1.pdf}

\begin{itemize}
\tightlist
\item
  Generate a boxplot visualizing the number of total victims, by type of
  location.
\end{itemize}

\begin{Shaded}
\begin{Highlighting}[]
\NormalTok{victims\_location\_mass\_shootings}\OtherTok{\textless{}{-}}\NormalTok{mass\_shootings }\SpecialCharTok{\%\textgreater{}\%} 
  \FunctionTok{group\_by}\NormalTok{(location\_type) }\SpecialCharTok{\%\textgreater{}\%} 
  \FunctionTok{ggplot}\NormalTok{(}\FunctionTok{aes}\NormalTok{(}\AttributeTok{x=}\NormalTok{location\_type,}\AttributeTok{y=}\NormalTok{total\_victims))}\SpecialCharTok{+}\FunctionTok{geom\_boxplot}\NormalTok{()}\SpecialCharTok{+}
\FunctionTok{labs}\NormalTok{(}\AttributeTok{title =} \StringTok{"Total Victims in Mass Shootings by Location "}\NormalTok{,}
       \AttributeTok{x =} \StringTok{"Location"}\NormalTok{, }\AttributeTok{y=}\StringTok{"Number of Victims"}\NormalTok{) }\SpecialCharTok{+}
  \FunctionTok{theme\_bw}\NormalTok{() }

\FunctionTok{print}\NormalTok{(victims\_location\_mass\_shootings)}
\end{Highlighting}
\end{Shaded}

\includegraphics{homework2_v2_files/figure-latex/unnamed-chunk-5-1.pdf}

\begin{itemize}
\tightlist
\item
  Redraw the same plot, but remove the Las Vegas Strip massacre from the
  dataset.
\end{itemize}

\begin{Shaded}
\begin{Highlighting}[]
\NormalTok{victims\_location\_mass\_shootings\_excl}\OtherTok{\textless{}{-}}
\NormalTok{  mass\_shootings }\SpecialCharTok{\%\textgreater{}\%} 
  \FunctionTok{filter}\NormalTok{(case }\SpecialCharTok{!=} \StringTok{"Las Vegas Strip massacre"}\NormalTok{) }\SpecialCharTok{\%\textgreater{}\%} 
  \FunctionTok{group\_by}\NormalTok{(location\_type) }\SpecialCharTok{\%\textgreater{}\%} 
  \FunctionTok{ggplot}\NormalTok{(}\FunctionTok{aes}\NormalTok{(}\AttributeTok{x=}\NormalTok{location\_type,}\AttributeTok{y=}\NormalTok{total\_victims))}\SpecialCharTok{+}\FunctionTok{geom\_boxplot}\NormalTok{()}\SpecialCharTok{+}
\FunctionTok{labs}\NormalTok{(}\AttributeTok{title =} \StringTok{"Total Victims in Mass Shootings by Location "}\NormalTok{,}
       \AttributeTok{x =} \StringTok{"Location"}\NormalTok{, }\AttributeTok{y=}\StringTok{"Number of Victims"}\NormalTok{) }\SpecialCharTok{+}
  \FunctionTok{theme\_bw}\NormalTok{()}

\FunctionTok{print}\NormalTok{(victims\_location\_mass\_shootings\_excl)}
\end{Highlighting}
\end{Shaded}

\includegraphics{homework2_v2_files/figure-latex/unnamed-chunk-6-1.pdf}

\hypertarget{more-open-ended-questions}{%
\subsubsection{More open-ended
questions}\label{more-open-ended-questions}}

Address the following questions. Generate appropriate figures/tables to
support your conclusions.

\begin{itemize}
\tightlist
\item
  How many white males with prior signs of mental illness initiated a
  mass shooting after 2000?
\end{itemize}

\begin{Shaded}
\begin{Highlighting}[]
\NormalTok{mass\_shootings }\SpecialCharTok{\%\textgreater{}\%}
  \FunctionTok{filter}\NormalTok{(year }\SpecialCharTok{\textgreater{}} \DecValTok{2000}\NormalTok{, race }\SpecialCharTok{==} \StringTok{"White"}\NormalTok{, prior\_mental\_illness }\SpecialCharTok{==}\StringTok{"Yes"}\NormalTok{) }\SpecialCharTok{\%\textgreater{}\%} 
  \FunctionTok{summarise}\NormalTok{(}\AttributeTok{count=}\FunctionTok{n}\NormalTok{())}
\end{Highlighting}
\end{Shaded}

\begin{verbatim}
## # A tibble: 1 x 1
##   count
##   <int>
## 1    23
\end{verbatim}

\begin{Shaded}
\begin{Highlighting}[]
\CommentTok{\#The number of white males with prior signs of mental illness initiated a mass shooting after 2000 is 23}
\end{Highlighting}
\end{Shaded}

\begin{itemize}
\tightlist
\item
  Which month of the year has the most mass shootings? Generate a bar
  chart sorted in chronological (natural) order (Jan-Feb-Mar- etc) to
  provide evidence of your answer.
\end{itemize}

\begin{Shaded}
\begin{Highlighting}[]
\NormalTok{month\_order}\OtherTok{\textless{}{-}}\FunctionTok{c}\NormalTok{(}\StringTok{"Jan"}\NormalTok{,}\StringTok{"Feb"}\NormalTok{,}\StringTok{"Mar"}\NormalTok{,}\StringTok{"Apr"}\NormalTok{,}\StringTok{"May"}\NormalTok{,}\StringTok{"Jun"}\NormalTok{,}\StringTok{"Jul"}\NormalTok{,}\StringTok{"Aug"}\NormalTok{,}\StringTok{"Sep"}\NormalTok{,}\StringTok{"Oct"}\NormalTok{,}\StringTok{"Nov"}\NormalTok{,}\StringTok{"Dec"}\NormalTok{)}

\NormalTok{mass\_shootings}\SpecialCharTok{\%\textgreater{}\%} 
  \FunctionTok{mutate}\NormalTok{(}\AttributeTok{month =} \FunctionTok{factor}\NormalTok{(month, }\AttributeTok{levels=}\NormalTok{month\_order)) }\SpecialCharTok{\%\textgreater{}\%} 
  \FunctionTok{group\_by}\NormalTok{(month) }\SpecialCharTok{\%\textgreater{}\%} 
  \FunctionTok{summarise}\NormalTok{(}\AttributeTok{count=}\FunctionTok{n}\NormalTok{()) }
\end{Highlighting}
\end{Shaded}

\begin{verbatim}
## # A tibble: 12 x 2
##    month count
##    <fct> <int>
##  1 Jan       7
##  2 Feb      13
##  3 Mar      12
##  4 Apr      11
##  5 May       8
##  6 Jun      12
##  7 Jul      10
##  8 Aug       8
##  9 Sep      10
## 10 Oct      11
## 11 Nov      12
## 12 Dec      11
\end{verbatim}

\begin{Shaded}
\begin{Highlighting}[]
\CommentTok{\#convert month into factor first which has category and then order}
\end{Highlighting}
\end{Shaded}

\begin{itemize}
\tightlist
\item
  How does the distribution of mass shooting fatalities differ between
  White and Black shooters? What about White and Latino shooters?
\end{itemize}

\begin{Shaded}
\begin{Highlighting}[]
\NormalTok{mass\_shootings}\SpecialCharTok{\%\textgreater{}\%} 
  \FunctionTok{filter}\NormalTok{(race }\SpecialCharTok{\%in\%} \FunctionTok{c}\NormalTok{(}\StringTok{"White"}\NormalTok{,}\StringTok{"Black"}\NormalTok{,}\StringTok{"Latino"}\NormalTok{)) }\SpecialCharTok{\%\textgreater{}\%} 
 \FunctionTok{group\_by}\NormalTok{(race) }\SpecialCharTok{\%\textgreater{}\%} 
  \FunctionTok{ggplot}\NormalTok{(}\FunctionTok{aes}\NormalTok{(}\AttributeTok{x=}\NormalTok{fatalities)) }\SpecialCharTok{+}
  \FunctionTok{geom\_histogram}\NormalTok{() }\SpecialCharTok{+}
  \FunctionTok{facet\_wrap}\NormalTok{(}\SpecialCharTok{\textasciitilde{}}\NormalTok{race)}
\end{Highlighting}
\end{Shaded}

\begin{verbatim}
## `stat_bin()` using `bins = 30`. Pick better value with `binwidth`.
\end{verbatim}

\includegraphics{homework2_v2_files/figure-latex/unnamed-chunk-9-1.pdf}

\hypertarget{very-open-ended}{%
\subsubsection{Very open-ended}\label{very-open-ended}}

\begin{itemize}
\tightlist
\item
  Are mass shootings with shooters suffering from mental illness
  different from mass shootings with no signs of mental illness in the
  shooter?
\end{itemize}

\begin{Shaded}
\begin{Highlighting}[]
\NormalTok{mass\_shootings}\SpecialCharTok{\%\textgreater{}\%} 
 \FunctionTok{group\_by}\NormalTok{(prior\_mental\_illness) }\SpecialCharTok{\%\textgreater{}\%} 
  \FunctionTok{summarize}\NormalTok{(}\AttributeTok{count=}\FunctionTok{n}\NormalTok{())}
\end{Highlighting}
\end{Shaded}

\begin{verbatim}
## # A tibble: 3 x 2
##   prior_mental_illness count
##   <chr>                <int>
## 1 No                      17
## 2 Yes                     62
## 3 <NA>                    46
\end{verbatim}

\begin{itemize}
\tightlist
\item
  Assess the relationship between mental illness and total victims,
  mental illness and location type, and the intersection of all three
  variables.
\end{itemize}

\begin{Shaded}
\begin{Highlighting}[]
\NormalTok{mass\_shootings}\SpecialCharTok{\%\textgreater{}\%} 
 \FunctionTok{group\_by}\NormalTok{(prior\_mental\_illness) }\SpecialCharTok{\%\textgreater{}\%} 
  \FunctionTok{drop\_na}\NormalTok{() }\SpecialCharTok{\%\textgreater{}\%} 
  \FunctionTok{ggplot}\NormalTok{(}\FunctionTok{aes}\NormalTok{(}\AttributeTok{x=}\NormalTok{prior\_mental\_illness,}\AttributeTok{y=}\NormalTok{total\_victims))}\SpecialCharTok{+}\FunctionTok{geom\_boxplot}\NormalTok{()}\SpecialCharTok{+}
\FunctionTok{labs}\NormalTok{(}\AttributeTok{title =} \StringTok{"Total Victims in Mass Shootings by Prior Mental Illness "}\NormalTok{,}
       \AttributeTok{x =} \StringTok{"Prior Mental Illness"}\NormalTok{, }\AttributeTok{y=}\StringTok{"Number of Victims"}\NormalTok{) }\SpecialCharTok{+}
  \FunctionTok{theme\_bw}\NormalTok{()}
\end{Highlighting}
\end{Shaded}

\includegraphics{homework2_v2_files/figure-latex/unnamed-chunk-11-1.pdf}

\begin{Shaded}
\begin{Highlighting}[]
\NormalTok{mass\_shootings}\SpecialCharTok{\%\textgreater{}\%} 
 \FunctionTok{group\_by}\NormalTok{(prior\_mental\_illness) }\SpecialCharTok{\%\textgreater{}\%} 
  \FunctionTok{drop\_na}\NormalTok{() }\SpecialCharTok{\%\textgreater{}\%} 
  \FunctionTok{ggplot}\NormalTok{(}\FunctionTok{aes}\NormalTok{(}\AttributeTok{x=}\NormalTok{prior\_mental\_illness,}\AttributeTok{fill=}\NormalTok{location\_type))}\SpecialCharTok{+}\FunctionTok{geom\_bar}\NormalTok{()}\SpecialCharTok{+}
\FunctionTok{labs}\NormalTok{(}\AttributeTok{title =} \StringTok{"Mental Illness vs Location Type "}\NormalTok{,}
       \AttributeTok{x =} \StringTok{"Prior Mental Illness"}\NormalTok{, }\AttributeTok{y=}\StringTok{"Location Type"}\NormalTok{) }\SpecialCharTok{+}
  \FunctionTok{theme\_bw}\NormalTok{()}
\end{Highlighting}
\end{Shaded}

\includegraphics{homework2_v2_files/figure-latex/unnamed-chunk-12-1.pdf}

\begin{Shaded}
\begin{Highlighting}[]
\NormalTok{mass\_shootings}\SpecialCharTok{\%\textgreater{}\%} 
 \FunctionTok{group\_by}\NormalTok{(prior\_mental\_illness,location\_type) }\SpecialCharTok{\%\textgreater{}\%} 
  \FunctionTok{drop\_na}\NormalTok{() }\SpecialCharTok{\%\textgreater{}\%} 
  \FunctionTok{ggplot}\NormalTok{(}\FunctionTok{aes}\NormalTok{(}\AttributeTok{x=}\NormalTok{location\_type,}\AttributeTok{y=}\NormalTok{total\_victims,}\AttributeTok{fill=}\NormalTok{prior\_mental\_illness))}\SpecialCharTok{+}\FunctionTok{geom\_bar}\NormalTok{(}\AttributeTok{stat=}\StringTok{"identity"}\NormalTok{,}\AttributeTok{position=}\StringTok{"dodge"}\NormalTok{)}\SpecialCharTok{+}
\FunctionTok{labs}\NormalTok{(}\AttributeTok{title =} \StringTok{"Relationship between Mental Illness and Location on Total Number of Victims  "}\NormalTok{,}
       \AttributeTok{x =} \StringTok{"Location Type"}\NormalTok{, }\AttributeTok{y=}\StringTok{"Total Victim"}\NormalTok{, }\AttributeTok{fill=}\StringTok{"Mental Illness"}\NormalTok{) }\SpecialCharTok{+}
  \FunctionTok{theme\_bw}\NormalTok{()}
\end{Highlighting}
\end{Shaded}

\includegraphics{homework2_v2_files/figure-latex/unnamed-chunk-13-1.pdf}

Make sure to provide a couple of sentences of written interpretation of
your tables/figures. Graphs and tables alone will not be sufficient to
answer this question.

\hypertarget{exploring-credit-card-fraud}{%
\section{Exploring credit card
fraud}\label{exploring-credit-card-fraud}}

We will be using a dataset with credit card transactions containing
legitimate and fraud transactions. Fraud is typically well below 1\% of
all transactions, so a naive model that predicts that all transactions
are legitimate and not fraudulent would have an accuracy of well over
99\%-- pretty good, no? (well, not quite as we will see later in the
course)

You can read more on credit card fraud on
\href{https://www.scirp.org/journal/paperinformation.aspx?paperid=105944}{Credit
Card Fraud Detection Using Weighted Support Vector Machine}

The dataset we will use consists of credit card transactions and it
includes information about each transaction including customer details,
the merchant and category of purchase, and whether or not the
transaction was a fraud.

\hypertarget{obtain-the-data-1}{%
\subsection{Obtain the data}\label{obtain-the-data-1}}

The dataset is too large to be hosted on Canvas or Github, so please
download it from dropbox
\url{https://www.dropbox.com/sh/q1yk8mmnbbrzavl/AAAxzRtIhag9Nc_hODafGV2ka?dl=0}
and save it in your \texttt{dsb} repo, under the \texttt{data} folder

\begin{verbatim}
## Rows: 671,028
## Columns: 14
## $ trans_date_trans_time <dttm> 2019-02-22 07:32:58, 2019-02-16 15:07:20, 2019-~
## $ trans_year            <dbl> 2019, 2019, 2019, 2019, 2019, 2019, 2019, 2020, ~
## $ category              <chr> "entertainment", "kids_pets", "personal_care", "~
## $ amt                   <dbl> 7.79, 3.89, 8.43, 40.00, 54.04, 95.61, 64.95, 3.~
## $ city                  <chr> "Veedersburg", "Holloway", "Arnold", "Apison", "~
## $ state                 <chr> "IN", "OH", "MO", "TN", "CO", "GA", "MN", "AL", ~
## $ lat                   <dbl> 40.1186, 40.0113, 38.4305, 35.0149, 39.4584, 32.~
## $ long                  <dbl> -87.2602, -80.9701, -90.3870, -85.0164, -106.385~
## $ city_pop              <dbl> 4049, 128, 35439, 3730, 277, 1841, 136, 190178, ~
## $ job                   <chr> "Development worker, community", "Child psychoth~
## $ dob                   <chr> "19/10/1959", "3/4/1946", "31/3/1985", "28/1/199~
## $ merch_lat             <dbl> 39.41679, 39.74585, 37.73078, 34.53277, 39.95244~
## $ merch_long            <dbl> -87.52619, -81.52477, -91.36875, -84.10676, -106~
## $ is_fraud              <dbl> 0, 0, 0, 0, 0, 0, 0, 0, 0, 0, 0, 0, 0, 0, 0, 0, ~
\end{verbatim}

The data dictionary is as follows

\begin{longtable}[]{@{}ll@{}}
\toprule()
column(variable) & description \\
\midrule()
\endhead
trans\_date\_trans\_time & Transaction DateTime \\
trans\_year & Transaction year \\
category & category of merchant \\
amt & amount of transaction \\
city & City of card holder \\
state & State of card holder \\
lat & Latitude location of purchase \\
long & Longitude location of purchase \\
city\_pop & card holder's city population \\
job & job of card holder \\
dob & date of birth of card holder \\
merch\_lat & Latitude Location of Merchant \\
merch\_long & Longitude Location of Merchant \\
is\_fraud & Whether Transaction is Fraud (1) or Not (0) \\
\bottomrule()
\end{longtable}

\begin{itemize}
\tightlist
\item
  In this dataset, how likely are fraudulent transactions? Generate a
  table that summarizes the number and frequency of fraudulent
  transactions per year.
\end{itemize}

\begin{Shaded}
\begin{Highlighting}[]
\NormalTok{card\_fraud }\SpecialCharTok{\%\textgreater{}\%}
  \FunctionTok{group\_by}\NormalTok{(trans\_year) }\SpecialCharTok{\%\textgreater{}\%} 
  \FunctionTok{summarise}\NormalTok{(}\AttributeTok{count=}\FunctionTok{n}\NormalTok{())}
\end{Highlighting}
\end{Shaded}

\begin{verbatim}
## # A tibble: 2 x 2
##   trans_year  count
##        <dbl>  <int>
## 1       2019 478646
## 2       2020 192382
\end{verbatim}

\begin{itemize}
\tightlist
\item
  How much money (in US\$ terms) are fraudulent transactions costing the
  company? Generate a table that summarizes the total amount of
  legitimate and fraudulent transactions per year and calculate the \%
  of fraudulent transactions, in US\$ terms.
\end{itemize}

\begin{Shaded}
\begin{Highlighting}[]
\NormalTok{fraudulent\_transaction\_cost}\OtherTok{\textless{}{-}}\NormalTok{card\_fraud }\SpecialCharTok{\%\textgreater{}\%}
  \FunctionTok{mutate}\NormalTok{(}\AttributeTok{is\_fraud =} \FunctionTok{case\_when}\NormalTok{(is\_fraud }\SpecialCharTok{==} \DecValTok{0} \SpecialCharTok{\textasciitilde{}} \StringTok{"Legitimate"}\NormalTok{, is\_fraud }\SpecialCharTok{==} \DecValTok{1} \SpecialCharTok{\textasciitilde{}} \StringTok{"Fraudulent"}\NormalTok{)) }\SpecialCharTok{\%\textgreater{}\%}
  \FunctionTok{group\_by}\NormalTok{(trans\_year, is\_fraud) }\SpecialCharTok{\%\textgreater{}\%}
  \FunctionTok{summarise}\NormalTok{(}\AttributeTok{total\_amount =} \FunctionTok{sum}\NormalTok{(amt),}\AttributeTok{.groups=}\StringTok{\textquotesingle{}keep\textquotesingle{}}\NormalTok{) }\SpecialCharTok{\%\textgreater{}\%}
  \FunctionTok{pivot\_wider}\NormalTok{(}\AttributeTok{names\_from =}\NormalTok{ is\_fraud, }\AttributeTok{values\_from =}\NormalTok{ total\_amount, }\AttributeTok{values\_fill =} \DecValTok{0}\NormalTok{) }\SpecialCharTok{\%\textgreater{}\%}
  \FunctionTok{mutate}\NormalTok{(}\AttributeTok{fraud\_percentage =}\NormalTok{ Fraudulent }\SpecialCharTok{/}\NormalTok{ (Legitimate }\SpecialCharTok{+}\NormalTok{ Fraudulent) }\SpecialCharTok{*} \DecValTok{100}\NormalTok{)}

\FunctionTok{print}\NormalTok{(fraudulent\_transaction\_cost)}
\end{Highlighting}
\end{Shaded}

\begin{verbatim}
## # A tibble: 2 x 4
## # Groups:   trans_year [2]
##   trans_year Fraudulent Legitimate fraud_percentage
##        <dbl>      <dbl>      <dbl>            <dbl>
## 1       2019   1423140.  32182901.             4.23
## 2       2020    651949.  12925914.             4.80
\end{verbatim}

\begin{itemize}
\tightlist
\item
  Generate a histogram that shows the distribution of amounts charged to
  credit card, both for legitimate and fraudulent accounts. Also, for
  both types of transactions, calculate some quick summary statistics.
\end{itemize}

\begin{Shaded}
\begin{Highlighting}[]
\NormalTok{summary\_table }\OtherTok{\textless{}{-}}\NormalTok{ card\_fraud }\SpecialCharTok{\%\textgreater{}\%}
  \FunctionTok{group\_by}\NormalTok{(trans\_year, is\_fraud) }\SpecialCharTok{\%\textgreater{}\%}
  \FunctionTok{summarize}\NormalTok{(}\AttributeTok{total\_amt =} \FunctionTok{sum}\NormalTok{(amt))}
\end{Highlighting}
\end{Shaded}

\begin{verbatim}
## `summarise()` has grouped output by 'trans_year'. You can override using the
## `.groups` argument.
\end{verbatim}

\begin{Shaded}
\begin{Highlighting}[]
\CommentTok{\# total fraud}
\NormalTok{fraud\_sum }\OtherTok{\textless{}{-}}\NormalTok{ summary\_table }\SpecialCharTok{\%\textgreater{}\%}
  \FunctionTok{filter}\NormalTok{(is\_fraud }\SpecialCharTok{==} \DecValTok{1}\NormalTok{) }\SpecialCharTok{\%\textgreater{}\%}
  \FunctionTok{mutate}\NormalTok{(}\AttributeTok{fraud\_amt =}\NormalTok{ total\_amt) }\SpecialCharTok{\%\textgreater{}\%}
  \FunctionTok{select}\NormalTok{(trans\_year, fraud\_amt)}

\CommentTok{\# total legit}
\NormalTok{legitimate\_sum }\OtherTok{\textless{}{-}}\NormalTok{ summary\_table }\SpecialCharTok{\%\textgreater{}\%}
  \FunctionTok{filter}\NormalTok{(is\_fraud }\SpecialCharTok{==} \DecValTok{0}\NormalTok{) }\SpecialCharTok{\%\textgreater{}\%}
  \FunctionTok{mutate}\NormalTok{(}\AttributeTok{legitimate\_amt =}\NormalTok{ total\_amt) }\SpecialCharTok{\%\textgreater{}\%}
  \FunctionTok{select}\NormalTok{(trans\_year, legitimate\_amt)}
  
\NormalTok{fraud\_percentage }\OtherTok{\textless{}{-}}\NormalTok{ fraud\_sum }\SpecialCharTok{\%\textgreater{}\%}
  \FunctionTok{left\_join}\NormalTok{(legitimate\_sum, }\AttributeTok{by =} \StringTok{"trans\_year"}\NormalTok{) }\SpecialCharTok{\%\textgreater{}\%}
  \FunctionTok{mutate}\NormalTok{(}\AttributeTok{percentage\_fraud =}\NormalTok{ (fraud\_amt }\SpecialCharTok{/}\NormalTok{ (fraud\_amt }\SpecialCharTok{+}\NormalTok{ legitimate\_amt)) }\SpecialCharTok{*} \DecValTok{100}\NormalTok{, }\AttributeTok{percentage\_legitimate =}\NormalTok{ (legitimate\_amt }\SpecialCharTok{/}\NormalTok{ (fraud\_amt }\SpecialCharTok{+}\NormalTok{ legitimate\_amt)) }\SpecialCharTok{*} \DecValTok{100}\NormalTok{) }\SpecialCharTok{\%\textgreater{}\%}
  \FunctionTok{select}\NormalTok{(trans\_year, fraud\_amt, legitimate\_amt, percentage\_fraud, percentage\_legitimate) }\SpecialCharTok{\%\textgreater{}\%} 
  
\FunctionTok{ggplot}\NormalTok{(}\FunctionTok{aes}\NormalTok{(}\AttributeTok{x =} \FunctionTok{factor}\NormalTok{(trans\_year), }\AttributeTok{y =}\NormalTok{ fraud\_amt }\SpecialCharTok{+}\NormalTok{ legitimate\_amt)) }\SpecialCharTok{+}
  \FunctionTok{geom\_col}\NormalTok{(}\FunctionTok{aes}\NormalTok{(}\AttributeTok{fill =} \FunctionTok{paste0}\NormalTok{(}\FunctionTok{sprintf}\NormalTok{(}\StringTok{"\%.1f"}\NormalTok{, }\FunctionTok{round}\NormalTok{(percentage\_fraud, }\DecValTok{1}\NormalTok{)), }\StringTok{"\% Fraud}\SpecialCharTok{\textbackslash{}n}\StringTok{"}\NormalTok{, }\FunctionTok{sprintf}\NormalTok{(}\StringTok{"\%.1f"}\NormalTok{, }\FunctionTok{round}\NormalTok{(percentage\_legitimate, }\DecValTok{1}\NormalTok{)), }\StringTok{"\% Legitimate"}\NormalTok{)), }\AttributeTok{width =} \FloatTok{0.8}\NormalTok{) }\SpecialCharTok{+}
  \FunctionTok{labs}\NormalTok{(}\AttributeTok{title =} \StringTok{"Total Amount per Year"}\NormalTok{, }\AttributeTok{x =} \StringTok{"Year"}\NormalTok{, }\AttributeTok{y =} \StringTok{"Total Amount"}\NormalTok{) }\SpecialCharTok{+}
  \FunctionTok{scale\_fill\_manual}\NormalTok{(}\AttributeTok{values =} \FunctionTok{c}\NormalTok{(}\StringTok{"grey"}\NormalTok{, }\StringTok{"black"}\NormalTok{), }\AttributeTok{name =} \StringTok{"Transaction Type"}\NormalTok{) }\SpecialCharTok{+}
  \FunctionTok{scale\_x\_discrete}\NormalTok{(}\AttributeTok{labels =} \FunctionTok{c}\NormalTok{(}\StringTok{"2019"}\NormalTok{, }\StringTok{"2020"}\NormalTok{)) }\SpecialCharTok{+}
  \FunctionTok{scale\_y\_continuous}\NormalTok{(}\AttributeTok{labels =}\NormalTok{ scales}\SpecialCharTok{::}\NormalTok{comma) }\SpecialCharTok{+}
  \FunctionTok{theme}\NormalTok{(}\AttributeTok{axis.text.x =} \FunctionTok{element\_text}\NormalTok{(}\AttributeTok{size =} \DecValTok{10}\NormalTok{, }\AttributeTok{vjust =} \FloatTok{0.5}\NormalTok{))}

\FunctionTok{print}\NormalTok{(fraud\_percentage)}
\end{Highlighting}
\end{Shaded}

\includegraphics{homework2_v2_files/figure-latex/unnamed-chunk-17-1.pdf}

\begin{itemize}
\tightlist
\item
  What types of purchases are most likely to be instances of fraud?
  Consider category of merchants and produce a bar chart that shows \%
  of total fraudulent transactions sorted in order.
\end{itemize}

\begin{Shaded}
\begin{Highlighting}[]
\CommentTok{\# Calculate the percentage of fraudulent transactions by merchant category}
\NormalTok{fraud\_by\_category }\OtherTok{\textless{}{-}}\NormalTok{ card\_fraud }\SpecialCharTok{\%\textgreater{}\%}
  \FunctionTok{filter}\NormalTok{(is\_fraud }\SpecialCharTok{==} \DecValTok{1}\NormalTok{) }\SpecialCharTok{\%\textgreater{}\%}
  \FunctionTok{count}\NormalTok{(category, }\AttributeTok{wt =}\NormalTok{ amt) }\SpecialCharTok{\%\textgreater{}\%}
  \FunctionTok{mutate}\NormalTok{(}\AttributeTok{percentage =}\NormalTok{ (n }\SpecialCharTok{/} \FunctionTok{sum}\NormalTok{(n)) }\SpecialCharTok{*} \DecValTok{100}\NormalTok{) }\SpecialCharTok{\%\textgreater{}\%}
  \FunctionTok{arrange}\NormalTok{(}\FunctionTok{desc}\NormalTok{(percentage))}

\CommentTok{\# Create a bar chart}
\NormalTok{bar\_chart }\OtherTok{\textless{}{-}} \FunctionTok{ggplot}\NormalTok{(fraud\_by\_category, }\FunctionTok{aes}\NormalTok{(}\AttributeTok{x =} \FunctionTok{reorder}\NormalTok{(category, percentage), }\AttributeTok{y =}\NormalTok{ percentage)) }\SpecialCharTok{+}
  \FunctionTok{geom\_bar}\NormalTok{(}\AttributeTok{stat =} \StringTok{"identity"}\NormalTok{) }\SpecialCharTok{+}
  \FunctionTok{labs}\NormalTok{(}\AttributeTok{title =} \StringTok{"Percentage of Fraudulent Transactions by Merchant Category"}\NormalTok{,}
       \AttributeTok{x =} \StringTok{"Merchant Category"}\NormalTok{, }\AttributeTok{y =} \StringTok{"Percentage of Fraudulent Transactions"}\NormalTok{) }\SpecialCharTok{+}
  \FunctionTok{theme}\NormalTok{(}\AttributeTok{axis.text.x =} \FunctionTok{element\_text}\NormalTok{(}\AttributeTok{angle =} \DecValTok{45}\NormalTok{, }\AttributeTok{hjust =} \DecValTok{1}\NormalTok{))}

\CommentTok{\# Print the bar chart}
\NormalTok{bar\_chart}
\end{Highlighting}
\end{Shaded}

\includegraphics{homework2_v2_files/figure-latex/unnamed-chunk-18-1.pdf}

\begin{itemize}
\tightlist
\item
  When is fraud more prevalent? Which days, months, hours? To create new
  variables to help you in your analysis, we use the \texttt{lubridate}
  package and the following code
\end{itemize}

\begin{verbatim}
mutate(
  date_only = lubridate::date(trans_date_trans_time),
  month_name = lubridate::month(trans_date_trans_time, label=TRUE),
  hour = lubridate::hour(trans_date_trans_time),
  weekday = lubridate::wday(trans_date_trans_time, label = TRUE)
  )
\end{verbatim}

\begin{itemize}
\tightlist
\item
  Are older customers significantly more likely to be victims of credit
  card fraud? To calculate a customer's age, we use the
  \texttt{lubridate} package and the following code
\end{itemize}

\begin{Shaded}
\begin{Highlighting}[]
\NormalTok{card\_fraud }\OtherTok{\textless{}{-}}\NormalTok{ card\_fraud }\SpecialCharTok{\%\textgreater{}\%} 
  \FunctionTok{mutate}\NormalTok{(}
  \AttributeTok{date\_only =}\NormalTok{ lubridate}\SpecialCharTok{::}\FunctionTok{date}\NormalTok{(trans\_date\_trans\_time),}
  \AttributeTok{month\_name =}\NormalTok{ lubridate}\SpecialCharTok{::}\FunctionTok{month}\NormalTok{(trans\_date\_trans\_time, }\AttributeTok{label=}\ConstantTok{TRUE}\NormalTok{),}
  \AttributeTok{hour =}\NormalTok{ lubridate}\SpecialCharTok{::}\FunctionTok{hour}\NormalTok{(trans\_date\_trans\_time),}
  \AttributeTok{weekday =}\NormalTok{ lubridate}\SpecialCharTok{::}\FunctionTok{wday}\NormalTok{(trans\_date\_trans\_time, }\AttributeTok{label =} \ConstantTok{TRUE}\NormalTok{)}
\NormalTok{  )}

\CommentTok{\#When is fraud more prevalent? }
\CommentTok{\#by date \#2019{-}01{-}18 has the most fraud}
\NormalTok{card\_fraud }\SpecialCharTok{\%\textgreater{}\%} 
  \FunctionTok{filter}\NormalTok{(is\_fraud }\SpecialCharTok{==} \DecValTok{1}\NormalTok{) }\SpecialCharTok{\%\textgreater{}\%} 
  \FunctionTok{group\_by}\NormalTok{(date\_only) }\SpecialCharTok{\%\textgreater{}\%}
  \FunctionTok{select}\NormalTok{(date\_only, amt) }\SpecialCharTok{\%\textgreater{}\%} 
  \FunctionTok{arrange}\NormalTok{(}\FunctionTok{desc}\NormalTok{(amt))}
\end{Highlighting}
\end{Shaded}

\begin{verbatim}
## # A tibble: 3,936 x 2
## # Groups:   date_only [493]
##    date_only    amt
##    <date>     <dbl>
##  1 2019-01-18 1334.
##  2 2019-12-08 1313.
##  3 2020-06-07 1313.
##  4 2019-12-20 1295.
##  5 2019-06-03 1288.
##  6 2019-08-31 1282.
##  7 2019-05-20 1262.
##  8 2019-02-15 1259.
##  9 2020-03-10 1258.
## 10 2019-10-05 1246.
## # i 3,926 more rows
\end{verbatim}

\#by months \#Mar and May have the most fraud card\_fraud
\%\textgreater\% filter(is\_fraud == 1) \%\textgreater\%
group\_by(month\_name) \%\textgreater\% summarise(count = n())
\%\textgreater\% arrange(desc(count))

\#by hours \#Fraud is likely to happen at hour 23 card\_fraud
\%\textgreater\% filter(is\_fraud == 1) \%\textgreater\% group\_by(hour)
\%\textgreater\% summarise(count = n()) \%\textgreater\%
arrange(desc(count)) \#by day \#Monday has the most fraud card\_fraud
\%\textgreater\% filter(is\_fraud == 1) \%\textgreater\%
group\_by(weekday) \%\textgreater\% summarise(count = n())
\%\textgreater\% arrange(desc(count))

'\,'\,'

\begin{itemize}
\tightlist
\item
  Is fraud related to distance? The distance between a card holder's
  home and the location of the transaction can be a feature that is
  related to fraud. To calculate distance, we need the latidue/longitude
  of card holders's home and the latitude/longitude of the transaction,
  and we will use the
  \href{https://en.wikipedia.org/wiki/Haversine_formula}{Haversine
  formula} to calculate distance. I adapted code to
  \href{https://www.geeksforgeeks.org/program-distance-two-points-earth/amp/}{calculate
  distance between two points on earth} which you can find below
\end{itemize}

\begin{Shaded}
\begin{Highlighting}[]
\CommentTok{\# distance between card holder\textquotesingle{}s home and transaction}
\CommentTok{\# code adapted from https://www.geeksforgeeks.org/program{-}distance{-}two{-}points{-}earth/amp/}


\NormalTok{card\_fraud }\OtherTok{\textless{}{-}}\NormalTok{ card\_fraud }\SpecialCharTok{\%\textgreater{}\%}
  \FunctionTok{mutate}\NormalTok{(}
        \CommentTok{\# convert latitude/longitude to radians}
    \AttributeTok{lat1\_radians =}\NormalTok{ lat }\SpecialCharTok{/} \FloatTok{57.29577951}\NormalTok{,}
    \AttributeTok{lat2\_radians =}\NormalTok{ merch\_lat }\SpecialCharTok{/} \FloatTok{57.29577951}\NormalTok{,}
    \AttributeTok{long1\_radians =}\NormalTok{ long }\SpecialCharTok{/} \FloatTok{57.29577951}\NormalTok{,}
    \AttributeTok{long2\_radians =}\NormalTok{ merch\_long }\SpecialCharTok{/} \FloatTok{57.29577951}\NormalTok{,}
    
    \CommentTok{\# calculate distance in miles}
    \AttributeTok{distance\_miles =} \FloatTok{3963.0} \SpecialCharTok{*} \FunctionTok{acos}\NormalTok{((}\FunctionTok{sin}\NormalTok{(lat1\_radians) }\SpecialCharTok{*} \FunctionTok{sin}\NormalTok{(lat2\_radians)) }\SpecialCharTok{+} \FunctionTok{cos}\NormalTok{(lat1\_radians) }\SpecialCharTok{*} \FunctionTok{cos}\NormalTok{(lat2\_radians) }\SpecialCharTok{*} \FunctionTok{cos}\NormalTok{(long2\_radians }\SpecialCharTok{{-}}\NormalTok{ long1\_radians)),}

    \CommentTok{\# calculate distance in km}
    \AttributeTok{distance\_km =} \FloatTok{6377.830272} \SpecialCharTok{*} \FunctionTok{acos}\NormalTok{((}\FunctionTok{sin}\NormalTok{(lat1\_radians) }\SpecialCharTok{*} \FunctionTok{sin}\NormalTok{(lat2\_radians)) }\SpecialCharTok{+} \FunctionTok{cos}\NormalTok{(lat1\_radians) }\SpecialCharTok{*} \FunctionTok{cos}\NormalTok{(lat2\_radians) }\SpecialCharTok{*} \FunctionTok{cos}\NormalTok{(long2\_radians }\SpecialCharTok{{-}}\NormalTok{ long1\_radians))}
\NormalTok{  )}
\end{Highlighting}
\end{Shaded}

Plot a boxplot or a violin plot that looks at the relationship of
distance and \texttt{is\_fraud}. Does distance seem to be a useful
feature in explaining fraud?

\hypertarget{exploring-sources-of-electricity-production-co2-emissions-and-gdp-per-capita.}{%
\section{Exploring sources of electricity production, CO2 emissions, and
GDP per
capita.}\label{exploring-sources-of-electricity-production-co2-emissions-and-gdp-per-capita.}}

There are many sources of data on how countries generate their
electricity and their CO2 emissions. I would like you to create three
graphs:

\hypertarget{a-stacked-area-chart-that-shows-how-your-own-country-generated-its-electricity-since-2000.}{%
\subsection{1. A stacked area chart that shows how your own country
generated its electricity since
2000.}\label{a-stacked-area-chart-that-shows-how-your-own-country-generated-its-electricity-since-2000.}}

You will use

\texttt{geom\_area(colour="grey90",\ alpha\ =\ 0.5,\ position\ =\ "fill")}

\hypertarget{a-scatter-plot-that-looks-at-how-co2-per-capita-and-gdp-per-capita-are-related}{%
\subsection{2. A scatter plot that looks at how CO2 per capita and GDP
per capita are
related}\label{a-scatter-plot-that-looks-at-how-co2-per-capita-and-gdp-per-capita-are-related}}

\hypertarget{a-scatter-plot-that-looks-at-how-electricity-usage-kwh-per-capitaday-gdp-per-capita-are-related}{%
\subsection{3. A scatter plot that looks at how electricity usage (kWh)
per capita/day GDP per capita are
related}\label{a-scatter-plot-that-looks-at-how-electricity-usage-kwh-per-capitaday-gdp-per-capita-are-related}}

We will get energy data from the Our World in Data website, and CO2 and
GDP per capita emissions from the World Bank, using the
\texttt{wbstats}package.

\begin{Shaded}
\begin{Highlighting}[]
\CommentTok{\# Download electricity data}
\NormalTok{url }\OtherTok{\textless{}{-}} \StringTok{"https://nyc3.digitaloceanspaces.com/owid{-}public/data/energy/owid{-}energy{-}data.csv"}

\NormalTok{energy }\OtherTok{\textless{}{-}} \FunctionTok{read\_csv}\NormalTok{(url) }\SpecialCharTok{\%\textgreater{}\%} 
  \FunctionTok{filter}\NormalTok{(year }\SpecialCharTok{\textgreater{}=} \DecValTok{1990}\NormalTok{) }\SpecialCharTok{\%\textgreater{}\%} 
  \FunctionTok{drop\_na}\NormalTok{(iso\_code) }\SpecialCharTok{\%\textgreater{}\%} 
  \FunctionTok{select}\NormalTok{(}\DecValTok{1}\SpecialCharTok{:}\DecValTok{3}\NormalTok{,}
         \AttributeTok{biofuel =}\NormalTok{ biofuel\_electricity,}
         \AttributeTok{coal =}\NormalTok{ coal\_electricity,}
         \AttributeTok{gas =}\NormalTok{ gas\_electricity,}
         \AttributeTok{hydro =}\NormalTok{ hydro\_electricity,}
         \AttributeTok{nuclear =}\NormalTok{ nuclear\_electricity,}
         \AttributeTok{oil =}\NormalTok{ oil\_electricity,}
         \AttributeTok{other\_renewable =}\NormalTok{ other\_renewable\_exc\_biofuel\_electricity,}
         \AttributeTok{solar =}\NormalTok{ solar\_electricity,}
         \AttributeTok{wind =}\NormalTok{ wind\_electricity, }
\NormalTok{         electricity\_demand,}
\NormalTok{         electricity\_generation,}
\NormalTok{         net\_elec\_imports,  }\CommentTok{\# Net electricity imports, measured in terawatt{-}hours}
\NormalTok{         energy\_per\_capita, }\CommentTok{\# Primary energy consumption per capita, measured in kilowatt{-}hours Calculated by Our World in Data based on BP Statistical Review of World Energy and EIA International Energy Data}
\NormalTok{         energy\_per\_gdp,    }\CommentTok{\# Energy consumption per unit of GDP. This is measured in kilowatt{-}hours per 2011 international{-}$.}
\NormalTok{         per\_capita\_electricity, }\CommentTok{\#  Electricity generation per capita, measured in kilowatt{-}hours}
\NormalTok{  ) }

\CommentTok{\# Download data for C02 emissions per capita https://data.worldbank.org/indicator/EN.ATM.CO2E.PC}
\NormalTok{co2\_percap }\OtherTok{\textless{}{-}} \FunctionTok{wb\_data}\NormalTok{(}\AttributeTok{country =} \StringTok{"countries\_only"}\NormalTok{, }
                      \AttributeTok{indicator =} \StringTok{"EN.ATM.CO2E.PC"}\NormalTok{, }
                      \AttributeTok{start\_date =} \DecValTok{1990}\NormalTok{, }
                      \AttributeTok{end\_date =} \DecValTok{2022}\NormalTok{,}
                      \AttributeTok{return\_wide=}\ConstantTok{FALSE}\NormalTok{) }\SpecialCharTok{\%\textgreater{}\%} 
  \FunctionTok{filter}\NormalTok{(}\SpecialCharTok{!}\FunctionTok{is.na}\NormalTok{(value)) }\SpecialCharTok{\%\textgreater{}\%} 
  \CommentTok{\#drop unwanted variables}
  \FunctionTok{select}\NormalTok{(}\SpecialCharTok{{-}}\FunctionTok{c}\NormalTok{(unit, obs\_status, footnote, last\_updated)) }\SpecialCharTok{\%\textgreater{}\%} 
  \FunctionTok{rename}\NormalTok{(}\AttributeTok{year =}\NormalTok{ date,}
         \AttributeTok{co2percap =}\NormalTok{ value)}


\CommentTok{\# Download data for GDP per capita  https://data.worldbank.org/indicator/NY.GDP.PCAP.PP.KD}
\NormalTok{gdp\_percap }\OtherTok{\textless{}{-}} \FunctionTok{wb\_data}\NormalTok{(}\AttributeTok{country =} \StringTok{"countries\_only"}\NormalTok{, }
                      \AttributeTok{indicator =} \StringTok{"NY.GDP.PCAP.PP.KD"}\NormalTok{, }
                      \AttributeTok{start\_date =} \DecValTok{1990}\NormalTok{, }
                      \AttributeTok{end\_date =} \DecValTok{2022}\NormalTok{,}
                      \AttributeTok{return\_wide=}\ConstantTok{FALSE}\NormalTok{) }\SpecialCharTok{\%\textgreater{}\%} 
  \FunctionTok{filter}\NormalTok{(}\SpecialCharTok{!}\FunctionTok{is.na}\NormalTok{(value)) }\SpecialCharTok{\%\textgreater{}\%} 
  \CommentTok{\#drop unwanted variables}
  \FunctionTok{select}\NormalTok{(}\SpecialCharTok{{-}}\FunctionTok{c}\NormalTok{(unit, obs\_status, footnote, last\_updated)) }\SpecialCharTok{\%\textgreater{}\%} 
  \FunctionTok{rename}\NormalTok{(}\AttributeTok{year =}\NormalTok{ date,}
         \AttributeTok{GDPpercap =}\NormalTok{ value)}
\end{Highlighting}
\end{Shaded}

Specific questions:

\begin{enumerate}
\def\labelenumi{\arabic{enumi}.}
\tightlist
\item
  How would you turn \texttt{energy} to long, tidy format?
\item
  You may need to join these data frames

  \begin{itemize}
  \tightlist
  \item
    Use \texttt{left\_join} from \texttt{dplyr} to
    \href{http://r4ds.had.co.nz/relational-data.html}{join the tables}
  \item
    To complete the merge, you need a unique \emph{key} to match
    observations between the data frames. Country names may not be
    consistent among the three dataframes, so please use the 3-digit ISO
    code for each country
  \item
    An aside: There is a great package called
    \href{https://github.com/vincentarelbundock/countrycode}{\texttt{countrycode}}
    that helps solve the problem of inconsistent country names (Is it
    UK? United Kingdom? Great Britain?). \texttt{countrycode()} takes as
    an input a country's name in a specific format and outputs it using
    whatever format you specify.
  \end{itemize}
\item
  Write a function that takes as input any country's name and returns
  all three graphs. You can use the \texttt{patchwork} package to
  arrange the three graphs as shown below
\end{enumerate}

\includegraphics{images/electricity-co2-gdp.png}

\hypertarget{deliverables}{%
\section{Deliverables}\label{deliverables}}

There is a lot of explanatory text, comments, etc. You do not need
these, so delete them and produce a stand-alone document that you could
share with someone. Knit the edited and completed R Markdown (qmd) file
as a Word or HTML document (use the ``Knit'' button at the top of the
script editor window) and upload it to Canvas. You must be comitting and
pushing your changes to your own Github repo as you go along.

\hypertarget{details}{%
\section{Details}\label{details}}

\begin{itemize}
\tightlist
\item
  Who did you collaborate with: TYPE NAMES HERE
\item
  Approximately how much time did you spend on this problem set: 8 hours
\item
  What, if anything, gave you the most trouble: trying to solve the
  errors
\end{itemize}

\textbf{Please seek out help when you need it,} and remember the
\href{https://dsb2023.netlify.app/syllabus/\#the-15-minute-rule}{15-minute
rule}. You know enough R (and have enough examples of code from class
and your readings) to be able to do this. If you get stuck, ask for help
from others, post a question on Slack-- and remember that I am here to
help too!

\begin{quote}
As a true test to yourself, do you understand the code you submitted and
are you able to explain it to someone else?
\end{quote}

\hypertarget{rubric}{%
\section{Rubric}\label{rubric}}

13/13: Problem set is 100\% completed. Every question was attempted and
answered, and most answers are correct. Code is well-documented (both
self-documented and with additional comments as necessary). Used
tidyverse, instead of base R. Graphs and tables are properly labelled.
Analysis is clear and easy to follow, either because graphs are labeled
clearly or you've written additional text to describe how you interpret
the output. Multiple Github commits. Work is exceptional. I will not
assign these often.

8/13: Problem set is 60--80\% complete and most answers are correct.
This is the expected level of performance. Solid effort. Hits all the
elements. No clear mistakes. Easy to follow (both the code and the
output). A few Github commits.

5/13: Problem set is less than 60\% complete and/or most answers are
incorrect. This indicates that you need to improve next time. I will
hopefully not assign these often. Displays minimal effort. Doesn't
complete all components. Code is poorly written and not documented. Uses
the same type of plot for each graph, or doesn't use plots appropriate
for the variables being analyzed. No Github commits.

\end{document}
